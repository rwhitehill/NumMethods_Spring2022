\documentclass[12pt,a4paper]{article}

\input{../preamble_files/packages}
\input{../preamble_files/figures}
\input{../preamble_files/references}
\input{../preamble_files/shortcuts}
\input{../preamble_files/listings}
\usepackage{physics}

\pagestyle{fancy}
\lhead{Richard Whitehill}
\chead{MATH 551 -- HW 3}
\rhead{04/04/22}
\cfoot{\thepage~of~\pageref{LastPage}}

\newcommand{\prob}[2]{\textbf{#1)} #2}

\setlength{\parskip}{\baselineskip}
\setlength{\parindent}{0pt}

\begin{document}

\prob{1}{Let $f(x) = \sqrt{x}$}

a) Find the absolute and relative condition numbers of $f$.

The absolute and relative condition numbers for a function $f$ are given as
\begin{align*}
    C\qty(x) = \abs{f'\qty(x) } \\
    \kappa\qty(x) = \abs{\frac{xf'\qty(x)}{f\qty(x) }} 
.\end{align*}

Since 
    $f = \sqrt{x}$, 
then 
    $f'\qty(x) = \frac{1}{2 \sqrt{x}}$,
and
\begin{align*}
    C\qty(x) &= \frac{1}{2\sqrt{x}} \\
    \kappa\qty(x) &= \frac{1}{2\sqrt{x}}\frac{x}{\sqrt{x}} = \frac{1}{2}
.\end{align*}

b) Where is $f$ well conditioned in an absolute sense? Where is $f$ well conditioned in a relative sense?

In an absolute sense, we can see that the absolute condition number is inversely proportional to the root of $x$, so it is not well conditioned near $x = 0$, but it becomes more well conditioned at larger values of $x$.

In a relative sense, the function is well conditioned everywhere since $\kappa\qty(x)$ is a half everywhere (and at zero the limit approaches a half as well).

c) Suppose $x = 10^{-17}$ is replaced by $\hat{x} = 10^{-16}$. Using the absolute condition number of $f$, how much of a change is expected in $f$ due to this change in the argument?

Recall $C\qty(x) = \frac{1}{2\sqrt{x}}$, so we expect
\begin{align*}
    \abs{\hat{y} - y} = C\qty(10^{-17})\abs{10^{-17} - 10^{-16}} \approx 1.423 \cdot 10^{-8}
.\end{align*}

\prob{2}{Lagrange interpolation}

a) Determine the Lagrange form of the interpolating polynomial for the data (2,1), (3,3), and (4,5).
What is the degree of this interpolating polynomial?

Recall that the $i^{\rm th}$ Lagrange polynomial for a data set with $n$ data points is given as
\begin{align*}
    L_i\qty(x) = \prod_{k \ne i}^{n} \frac{x-x_k}{x_i-x_k}
.\end{align*}

Hence we have the Lagrangian basis for the given data points:
\begin{align*}
    L_0\qty(x) &= \frac{\qty(x-3)\qty(x-4)}{\qty(2-3)\qty(2-4)} = \frac{\qty(x-3)\qty(x-4)}{2} \\
    L_1\qty(x) &= -\qty(x-2)\qty(x-4) \\
    L_2\qty(x) &= \frac{\qty(x-2)\qty(x-3)}{2}
.\end{align*}

From the Lagrangian basis, we can determine an interpolating polynomial as
\begin{align*}
    p\qty(x) = \sum_{i=0}^{n} y_iL_i\qty(x) 
.\end{align*}

For this problem, we then have
\begin{align*}
    p\qty(x) &= 1\qty[\frac{\qty(x-3)\qty(x-4)}{2}] + 3\qty[-\qty(x-2)\qty(x-4)] + 5\qty[\frac{\qty(x-2)\qty(x-3)}{2}]  \\
    &= \frac{1}{2}\qty(x-3)\qty(x-4) - 3\qty(x-2)\qty(x-4)  + \frac{5}{2}\qty(x-2)\qty(x-3) 
.\end{align*}

Observe that the interpolating polynomial is of degree 2.

b) Determine the a Lagrange form of the interpolating polynomial for the data (2,2), (3,3) and (4,5).
What is the degree of this interpolating polynomial?

Since the $x$-values are the same as in part (a), the Lagrangian basis is the same, and the interpolating polynomial is slightly modified:
\begin{align*}
    p\qty(x) &= \frac{2}{2}\qty(x-3)\qty(x-4) - 3\qty(x-2)\qty(x-4)  + \frac{5}{2}\qty(x-2)\qty(x-3) \\
    &= \qty(x-3)\qty(x-4) - 3\qty(x-2)\qty(x-4)  + \frac{5}{2}\qty(x-2)\qty(x-3) 
.\end{align*}

Notice that the degree of the polynomial is of degree 2 as in part (a).

c) Based on your observation, is the interpolating polynomial always of degree $N$ for $N+1$ data points? 
Explain why.

Yes, the interpolating polynomial is always by construction of degree $n$ when we have $n+1$ data points.


\end{document}
