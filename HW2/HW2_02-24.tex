\documentclass[12pt,a4paper]{article}

\input{../preamble_files/packages}
\input{../preamble_files/figures}
\input{../preamble_files/references}
\input{../preamble_files/shortcuts}
\input{../preamble_files/listings}

\pagestyle{fancy}
\lhead{Richard Whitehill}
\chead{MATH 551 -- HW 2}
\rhead{02/24/22}
\cfoot{\thepage~of~\pageref{LastPage}}

\newcommand{\prob}[2]{\textbf{#1)} #2}

\setlength{\parskip}{\baselineskip}
\setlength{\parindent}{0pt}

\begin{document}

\prob{1}{}

(a) Write the equation for the tangent line to $y = f(x)$ at $x = p$.

$\rightarrow$ The equation of a line passing through $p,f(p)$ with slope $m$ is given as
\begin{align*}
y-f(p) = m(x-p)
\end{align*}
Since the line is tangent to the curve at $x = p$, the slope $m = f'(p)$. Hence,
\begin{align*}
y - f(p) = f'(p)(x-p) \Rightarrow y = f'(p)x + [f(p) - pf'(p)]
\end{align*}

(b) Solve for the $x$-intercept of the line in equation (a).

The $x$-intercept is defined as the value of $x$ such that $y = 0$. We see that
\begin{align*}
0 = f'(p)x + [f(p) - pf'(p)] \Rightarrow x = \frac{pf'(p)-f(p)}{f'(p)} = p - \frac{f(p)}{f'(p)}
\end{align*}
assuming that $f'(p) \not= 0$.

(c) Write the equation for the line that intersects the curve $y = f(x)$ at $x = p$ and $x = q$.

The slope of the line passing through $(p,f(p))$ and $(q,f(q))$ is 
\begin{align*}
m = \frac{f(p)-f(q)}{p-q}
\end{align*}
so we see
\begin{align*}
y - f(p) = \frac{f(p)-f(q)}{p-q}(x-p) \Rightarrow y = \frac{f(p)-f(q)}{p-q}(x-p) + f(p)
\end{align*}

(d) Solve for the $x$-intercept of the line in equation (c).

We can solve for the $x$-intercept for the equation in part (c) as we did for that in part (b):
\begin{align*}
0 &= \frac{f(p)-f(q)}{p-q}(x-p) + f(p) \\
&\frac{f(p)-f(q)}{p-q}(x-p) = -f(p) \\
x-p &= -f(p)\frac{p-q}{f(p)-f(q)} \\
x &= p - \frac{f(p)(p-q)}{f(p)-f(q)}
\end{align*}

\prob{2}{Starting with (0,1), perform two iterations of Newton iteration on the following systems of nonlinear equations
\begin{align*}
\begin{cases}
4x_1^2 - x_2^2 = 0 \\
4x_1x_2^2 - x_1 = 1
\end{cases}
\end{align*}
}

It has been shown that for the system
\begin{align*}
\begin{cases}
f_1(x_1,x_2) = 0 \\
f_2(x_1,x_2) = 0
\end{cases}
\end{align*}
we can find successive approximations to the root as follows:
\begin{align*}
\vec{x}_{n+1} = \vec{x}_{n} - J^{-1}f(\vec{x}_{n})
\end{align*}
where 
\begin{align*}
\vec{x}_n = \begin{pmatrix}
x_{1}^{(n)} \\ x_{2}^{(n)}
\end{pmatrix} 
&\quad
f(\vec{x}_{n}) = \begin{pmatrix}
f_1(x_1^{(n)},x_2^{(n)}) \\ f_2(x_1^{(n)},x_2^{(n)})
\end{pmatrix}
\\
J(x_1,x_2) &= \begin{pmatrix}
\partial f_1/\partial x_1 & \partial f_1/\partial x_2 \\
\partial f_2/\partial x_1 & \partial f_2/\partial x_2
\end{pmatrix}
\end{align*}
For our problem $f_1 = 4x_1^2 - x_2^2$ and $f_2 = 4x_1x_2^2 - x_1 - 1$, so Newton iteration gives
\begin{align*}
\begin{pmatrix}
x_{1}^{(n+1)} \\ x_{2}^{(n+1)}
\end{pmatrix} 
=
\begin{pmatrix}
x_{1}^{(n)} \\ x_{2}^{(n)}
\end{pmatrix} 
-
\begin{pmatrix}
8x_1^{(n)} & -2x_2^{(n)} \\
4(x_2^{(n)})^2 - 1 & 8x_1^{(n)}x_2^{(n)}
\end{pmatrix}^{-1}
\begin{pmatrix}
4(x_1^{(n)})^2 - (x_2^{(n)})^2 \\ 4x_1^{(n)}(x_2^{(n)})^2 - x_1^{(n)} - 1
\end{pmatrix}
\end{align*}
If we begin with $x_1^{(0)} = 0$ and $x_2^{(0)} = 1$, then
\begin{align*}
\begin{pmatrix}
x_{1}^{(1)} \\ x_{2}^{(1)}
\end{pmatrix} 
=
\begin{pmatrix}
0 \\ 1
\end{pmatrix} 
-
\begin{pmatrix}
0 & -2 \\
3 & 0
\end{pmatrix}^{-1}
\begin{pmatrix}
-1 \\ -1
\end{pmatrix}
=
\begin{pmatrix}
0 \\ 1
\end{pmatrix} 
-
\begin{pmatrix}
-1/3 \\ 1/2
\end{pmatrix} 
=
\begin{pmatrix}
1/3 \\ 1/2
\end{pmatrix}
\end{align*}
Repeating for a second time we see
\begin{align*}
\begin{pmatrix}
x_{1}^{(2)} \\ x_{2}^{(2)}
\end{pmatrix} 
=
\begin{pmatrix}
1/3 \\ 1/2
\end{pmatrix} 
-
\begin{pmatrix}
8/3 & -1 \\
0 & 4/3
\end{pmatrix}^{-1}
\begin{pmatrix}
7/36 \\ -1
\end{pmatrix}
=
\begin{pmatrix}
1/3 \\ 1/2
\end{pmatrix} 
-
\begin{pmatrix}
-5/24 \\ -3/4
\end{pmatrix} 
=
\begin{pmatrix}
13/24 \\ 5/4
\end{pmatrix}
\end{align*}

\prob{3}{Write a program to compute the square root of $m$ by bisection method when $m = 7,~13,~17$. The interval $[a,b]$ where the square root is located is $[s,s+1]$, where $s$ is an integer and $s^2 < m < (s+1)^2$. Please find the approximate solution with 7 iteration steps, i.e. find $c_0,c_1,\hdots,c_7$.}

\end{document}
